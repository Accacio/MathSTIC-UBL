% !TeX spellcheck = fr_FR
\begin{otherlanguage}{french}
\addchap{Résumé Étendú}
\epigraph{Il vaut mieux faire que dire.}{Alfred de \textsc{Musset}}
\minitoc%
 %----------------------------------------------------------------------------------------------
Afin de garantir le respect des chartes graphiques des établissements et la cohérence du support, les  logos, les noms des établissements, le numéro de l’école doctorale, le nom de l’école doctorale, l’image de fond et le titre: Thèse de Doctorat ne sont pas modifiables.
%----------------------------------------------------------------------------------------------
\addsec{Instructions sur les paramètres modifiables}

\begin{enumerate}[label= (\arabic*)]
	\item Indiquer dans le cas d’une co-tutelle des 2 établissements de délivrance du diplôme (Docteur de établissement 1 Docteur de établissement 2)
	\item Si la thèse est en co-tutelle, mettre le logo des 2 établissements
	\item Indiquer la spécialité
	\item Attention: le prénom doit être en minuscules (Jean) et le \textsc{Nom} en majuscules (\textsc{Brittef})
	\item Donner le titre complet de la thèse, éventuellement le sous titre, si nécessaire sur plusieurs lignes
	\item Indiquer la date et le lieu de soutenance de la thèse
	\item Indiquer le nom du (ou des) laboratoire (s) dans le(s)quel(s) le travail de thèse a été effectué, indiquer aussi si souhaité le nom de la (les) faculté(s) (UFR, école(s), Institut(s), Centre(s)\ldots), son 	(leurs) adresse(s)\ldots
    \item Indiquer le N\textordmasculine{} de thèse, si cela est opportun, ou faire disparaitre cet item de la couverture
	\item Indiquer le Prénom en minuscules et le \textsc{Nom} en majuscules, le titre de la personne et
	l’établissement dans lequel il effectue sa recherche. Exemples:
	\begin{itemize}
		\item Professeur, Université d’Angers
		\item Chercheur, CNRS, École Centrale de Nantes
		\item Professeur d’université –- Praticien Hospitalier, Université Paris V
		\item Maitre de conférences, Oniris
		\item Chargé de recherche, \textsc{inserm}, \textsc{hdr}, Université de Tours
	\end{itemize}
\end{enumerate}

S’il n’y a pas de co-direction, faire disparaitre cet item de la couverture.

%----------------------------------------------------------------------------------------------
\addsec{1\textsuperscript{ère} de couverture}

\addsubsec{Informations obligatoires}
\begin{itemize}
	\item Le nom de l'établissement qui délivre le doctorat et le nom de l’école doctorale.
	Dans le cas d’une cotutelle internationale de thèse, mentionner le nom de chacun des
	établissements.
	\item  L'unité de recherche.
	\item Le type de doctorat.
	\item Le champ disciplinaire dans lequel est soutenue la thèse.
	\item Le titre de la thèse ou l’intitulé des principaux travaux (le choix des mots du titre est	particulièrement important, tous ces mots étant indexés systématiquement dans les catalogues).
	\item Les prénoms (en minuscules) et noms (en majuscules) de l'auteur.
	La règle administrative veut que soit utilisé d'abord le nom patronymique, suivi éventuellement du 	nom d’usage, qu’il résulte du mariage ou de la filiation. Les deux noms sont indexés et interrogeables dans les catalogues et bases de signalement des thèses.
	\item Les mentions \enquote{épouse}, \enquote{époux} \enquote{dit}  ou \enquote{née} ne doivent pas être utilisées. Pour qu'il n'y 	ait pas de confusion possible entre les noms et prénoms de l'auteur, tous les noms doivent apparaître en majuscules.
	\item Les prénoms et noms du directeur de recherche.
	\item S'il y a deux directeurs, mentionner en premier le directeur principal. 
        Pour les thèses qui sont soutenues dans le cadre d'une cotutelle internationale, utiliser une barre oblique \enquote{/} pour séparer les deux directeurs de thèse.
	\item La tomaison éventuelle.
	\item La date de soutenance si elle est définitive.
	\item La composition du jury.
	\item Les noms et prénoms des membres du jury selon les mêmes règles que celles décrites précédemment, avec indication de la fonction de la personne et du rôle au sein du jury. 
\end{itemize}

N.B.: La page de titre est la première page du document. 
Elle est comptabilisée dans la pagination mais n'est jamais numérotée\ldots

%----------------------------------------------------------------------------------------------
\addsec{4\textsuperscript{e} de couverture}

Informations obligatoires (à inscrire également sur le bordereau d’enregistrement national):


\textbf{Le résumé en français}: Composé au maximum 1700 caractères, espaces compris, il est précis et permet de comprendre comment le sujet est abordé. Tous les mots du résumé sont indexés et peuvent servir à la recherche.

\textbf{Les mots clés en français}: Ils sont choisis par l’étudiant en accord avec son directeur de thèse, en fonction de la terminologie en vigueur dans sa discipline. La bibliothèque de l’établissement se servira les mots choisis pour définir l’indexation Rameau, langage en usage dans les catalogues collectifs.

\textbf{Le titre de la thèse en français}:
Le choix des mots du titre est particulièrement important, tous ces mots étant indexés systématiquement dans les catalogues.

\textbf{Le titre en anglais}:
Il sert au signalement de la thèse dans des bases de données internationales. Tous les mots sont indexés et peuvent servir à la recherche.

\textbf{Abstract}:
Composé au maximum de 1700 caractères, espaces compris, il est la transposition fidèle en anglais du résumé français, sans être une traduction littérale pure.
\emph{Dans le cas d’une thèse en cotutelle internationale, si la langue de la thèse n’est pas le français, un résumé substantiel en français est requis en plus des résumés prévus ci-dessus. Ce condensé, d'une dizaine de pages, est alors placé avant la table des matières.
}
\textbf{Les mots-clés en anglais}:
Ils servent au signalement de la thèse dans des bases de données internationales et au moissonnage OAI\@.


\begin{figure}[!htb]
	\centering
	\begin{tikzpicture}[scale=0.45,
	align=center,node distance=0.2cm]
	every node/.style={scale=0.45}
	
	
	\node[ scale=0.45,font=\Huge] (TAB) at (0,0) {%
		\centering%\begin{table}[]
		
		\begin{tabular}{@{}|c|c|c|c|c|c|c|c|c|@{}}
		\toprule
		& $T_{5}$ & $T_{6}$ & $T_{7}$ & $T_{8}$ & $T_{9}$ & $T_{10}$ & $T_{11}$ & $T_{12}$ \\ \midrule
		$T_{1}$ &         &         &         &         &         &          & 1        &          \\ \midrule
		$T_{2}$ &         &         & 1       &         &         &          &          &          \\ \midrule
		$T_{3}$ & 1       &         &         &         &         &          &          &          \\ \midrule
		$T_{4}$ &         &         &         & 1       &         &          &          &          \\ \bottomrule
		\end{tabular}%
		
	};
	
	\node[below = of TAB.south, scale=0.45,font=\Huge, node distance = 0.5cm, anchor = north] (TAB2) {%
		\centering%\begin{table}[]
		\begin{tabular}{@{}|c|c|c|c|c|c|c|c|c|@{}}
		\toprule
		& $T_{1}$ & $T_{2}$ & $T_{3}$ & $T_{4}$ & $T_{9}$ & $T_{10}$ & $T_{11}$ & $T_{12}$ \\ \midrule
		$T_{5}$ &         &         & 1       &         &         &          &          &          \\ \midrule
		$T_{6}$ &         &         &         &         &         &          &          & 1        \\ \midrule
		$T_{7}$ &         &         &         &         & 1       &          &          &          \\ \midrule
		$T_{8}$ &         &         &         & 1       &         &          &          &          \\ \bottomrule
		\end{tabular}%
	};
	
	\node[below = of TAB2.south, scale=0.45,font=\Huge, node distance = 0.5cm, anchor = north] (TAB3)  {%
		\centering%\begin{table}[]
		\begin{tabular}{@{}|c|c|c|c|c|c|c|c|c|@{}}
		\toprule
		& $T_{1}$ & $T_{2}$ & $T_{3}$ & $T_{4}$ & $T_{5}$ & $T_{6}$ & $T_{7}$ & $T_{8}$ \\ \midrule
		$T_{9}$ &         &    1    &         &         &         &          &          &          \\ \midrule
		$T_{10}$ &         &         &         &         &         &          &     1    &          \\ \midrule
		$T_{11}$ &         &      1  &         &         &         &          &          &          \\ \midrule
		$T_{12}$ &   1     &         &         &         &         &          &          &          \\ \bottomrule
		\end{tabular}%
	};
	
	\node[draw,thick, node distance = 2.5cm,left = of TAB2] (FOLDS) {3-fold};
	\node[node distance = 2.5cm,left = of FOLDS] (T) {$\mathcal{T}$};
	\draw[thick,->,>=stealth, line width=1.0pt] (FOLDS.north) |- node[name=yes,anchor=south west] {Fold-1} (TAB);
	\draw[thick,->,>=stealth, line width=1.0pt] (FOLDS.south) |- node[name=yes,anchor=south west] {Fold-3} (TAB3);
	\draw[thick,->,>=stealth, line width=1.0pt] (FOLDS.east) -- node[name=yes,anchor=south] {Fold-2} (TAB2);
	\draw[thick,->,>=stealth, line width=1.0pt] (T.east) -- node[name=yes,anchor=south] {DTW} (FOLDS);
	\node[above = of TAB, node distance = 0.1cm] (R2) {True 1-NN of T};
	\end{tikzpicture}
	\caption[A toy example of the ground-truth construction.]{A toy example of the ground-truth construction.}%
	\label{fig:gt-dtw}
\end{figure}

\begin{algorithm}
	\caption{Calculate $y = x^n$}
	\begin{algorithmic}
		\REQUIRE $n \geq 0 \vee x \neq 0$
		\ENSURE $y = x^n$
		\STATE $y \leftarrow 1$
		\IF{$n < 0$}
		\STATE $X \leftarrow 1 / x$
		\STATE $N \leftarrow -n$
		\ELSE
		\STATE $X \leftarrow x$
		\STATE $N \leftarrow n$
		\ENDIF
		\WHILE{$N \neq 0$}
		\IF{$N$ is even}
		\STATE $X \leftarrow X \times X$
		\STATE $N \leftarrow N / 2$
		\ELSE[$N$ is odd]
		\STATE $y \leftarrow y \times X$
		\STATE $N \leftarrow N - 1$
		\ENDIF
		\ENDWHILE
	\end{algorithmic}
\end{algorithm}

\textbf{L’intitulé et l'adresse de l'unité ou du laboratoire de rattachement}:
S’ils ne figurent pas en page de titre, en respectant les formes prescrites par l’établissement de soutenance.


\begin{table}
\caption{Les effets des traitements X et Y sur les quatre groupes d'étudeThe effects of treatments X and Y on the four groups studied.}%
\label{tab:treatmentsf}
\centering
\begin{tabular}{l l l}
\toprule
Groups & Treatment X & Treatment Y \\
\midrule
1 & 0.2 & 0.8\\
2 & 0.17 & 0.7\\
3 & 0.24 & 0.75\\
4 & 0.68 & 0.3\\
\bottomrule\\
\end{tabular}
\end{table}

\end{otherlanguage}
